\documentclass[8pt,a4paper]{article}

\usepackage[a4paper,margin=2.2cm]{geometry}
\usepackage{graphicx}
\usepackage{amsmath,amssymb}
\usepackage{booktabs}
\usepackage{enumitem}
\usepackage{hyperref}
\usepackage{xcolor}


\title{\textbf{OptiView Pro: UX/UI Handout}\\
Linked Dual-Space Visualization for 4-Objective Pareto Fronts\\
with ML-Driven Design Space Exploration}
\author{Final Integrated Solutio\\
Prepared by UX/Visual Interaction Design Lead}
\date{November 24, 2025}

\begin{document}
\maketitle

\section{Overview \& Core Design Philosophy}
OptiView Pro supports decision-making in a 4-objective multi-objective optimization (MOO) problem where
three objectives are continuous $(f_1,f_2,f_3)$ and one objective is binary/discrete $(f_4\in\{0,1\})$.
In addition, users explore the \emph{design/feature space} before objectives:
a 2D input plane $(x_1,x_2)$ with a PyTorch binary classifier producing a probability field
$P(\text{class}=1\mid x_1,x_2)$ and a decision boundary at $0.5$.

\medskip
\noindent\textbf{Design principle: Linked Dual-Space Understanding.}
The interface connects \emph{causes} (design space, ML predictions) to \emph{effects} (objective space, Pareto optimality)
through tightly linked views, enabling intuitive discovery and confident selection.

\medskip
\noindent\textbf{Pareto emphasis principle.}
Pareto points are always visually dominant via size/opacity/outline separation, while still grounded in the full feasible set.

\medskip
\noindent\textbf{Clutter control refinement.}
All cross-view synchronization remains on hover/selection, while \emph{optional linking-line cues} can be toggled on/off
to keep the workspace clean.

\section{Data \& System Context}
\begin{itemize}[leftmargin=2em]
  \item \textbf{Dataset A (Feasible set):} all feasible points with fields
  $(x_1,x_2,f_1,f_2,f_3,f_4)$; potentially large.
  \item \textbf{Dataset B (Pareto set):} nondominated subset with the same fields.
  \item \textbf{ML model output:} probability grid $Z(x_1,x_2)=P(\text{class}=1\mid x_1,x_2)$,
  derived from a 2D mesh and PyTorch forward pass.
\end{itemize}

\section{Primary User Goals}
\begin{enumerate}[leftmargin=2.2em]
  \item Identify Pareto-optimal solutions quickly.
  \item Understand trade-offs among $f_1,f_2,f_3$ and how $f_4$ changes the front.
  \item Relate feature-space regions to objective-space performance.
  \item Filter, shortlist, and compare candidates for final decisions.
\end{enumerate}

\section{Dashboard Layout (Final)}
The UI uses a split dashboard with linked panels:

\begin{itemize}[leftmargin=2em]
  \item \textbf{Top-Left: Design Space (Input \& ML).}
  2D probability heatmap on $(x_1,x_2)$ with decision boundary and point overlays.
  \item \textbf{Top-Right: Objective Space (Pareto \& Feasible Context).}
  Two faceted 3D scatter views split by $f_4$:
  \emph{Panel A}: $f_4=0$, \emph{Panel B}: $f_4=1$.
  \item \textbf{Bottom-Left (Support): Parallel Coordinates and/or SPLOM.}
  Multi-dimensional trade-off reading and brushing across all four objectives.
  \item \textbf{Bottom-Right: Details \& Inspector.}
  Exact values for hovered/selected points and pinned comparison table.
  \item \textbf{Right Sidebar: Controls \& Filters.}
  Global view toggles, opacity sliders, range filters, and linking controls.
\end{itemize}

\section{Panel Encodings \& Interactions}

\subsection{Design Space Panel (2D Heatmap)}
\textbf{Visual encoding:}
\begin{itemize}[leftmargin=2em]
  \item Heatmap color $\rightarrow Z(x_1,x_2)$ (probability of class 1).
  \item Decision boundary $\rightarrow$ thick contour at $Z=0.5$ (plus optional contours at $0.1,0.9$).
  \item Feasible points $\rightarrow$ tiny, low-opacity dots.
  \item Pareto points $\rightarrow$ larger, saturated markers with outline/glow.
  \item Binary objective $f_4$ $\rightarrow$ marker \emph{shape} (e.g., circle vs triangle) or stroke style.
\end{itemize}

\textbf{Key interactions:}
\begin{itemize}[leftmargin=2em]
  \item Hover a point $\Rightarrow$ highlight same point in objective space and tradeoff views.
  \item Lasso/box select region $\Rightarrow$ create a working set linked to all views.
  \item Probe cursor on grid $\Rightarrow$ show $Z(x_1,x_2)$ and nearest point objectives (optional).
  \item Zoom/pan for local inspection.
\end{itemize}

\subsection{Objective Space Panel (Faceted 3D)}
\textbf{Visual encoding:}
\begin{itemize}[leftmargin=2em]
  \item Axes: $(f_1,f_2,f_3)$ in 3D.
  \item Feasible cloud $\rightarrow$ very faint points or density volumes.
  \item Pareto front $\rightarrow$ bold points on top (size + glow).
  \item Faceting by $f_4$ $\rightarrow$ two side-by-side panels for direct comparison.
\end{itemize}

\textbf{Key interactions:}
\begin{itemize}[leftmargin=2em]
  \item Rotate/pan/zoom 3D.
  \item Hover a Pareto point $\Rightarrow$ ring highlight at its $(x_1,x_2)$ in heatmap.
  \item Brush-select in 3D $\Rightarrow$ filters heatmap and tradeoff views.
\end{itemize}

\subsection{Tradeoff View (Parallel Coordinates / SPLOM)}
\textbf{Purpose:} reduce 3D occlusion and enable precise multi-criteria filtering.

\textbf{Encoding:}
\begin{itemize}[leftmargin=2em]
  \item One axis per objective $f_1,f_2,f_3,f_4$ (binary shown as two categorical levels).
  \item Feasible set $\rightarrow$ thin, ghosted lines / density bands.
  \item Pareto set $\rightarrow$ thicker, saturated lines.
\end{itemize}

\textbf{Interactions:}
\begin{itemize}[leftmargin=2em]
  \item Range brushing on any axis $\Rightarrow$ linked filtering everywhere.
  \item Axis reordering to reveal correlations.
\end{itemize}

\subsection{Inspector Panel}
\begin{itemize}[leftmargin=2em]
  \item Shows $(x_1,x_2)$, $Z(x_1,x_2)$, predicted class, and $(f_1,f_2,f_3,f_4)$.
  \item \textbf{Pin/Compare:} users can pin multiple candidates and compare deltas.
\end{itemize}

\subsection{Controls \& Filters Sidebar}
\textbf{Core controls:}
\begin{itemize}[leftmargin=2em]
  \item Toggle: show/hide feasible cloud.
  \item Slider: feasible cloud opacity.
  \item Toggle: density vs sampled feasible points.
  \item Filters: objective range sliders and/or categorical $f_4$ filter.
  \item Toggle: split/combined view for $f_4$ (default split).
  \item \textbf{Toggle: show linking line}. When off, highlighting remains but lines are hidden.
  \item Toggle: show sampled points on heatmap.
\end{itemize}

\section{Pareto Emphasis Rules (Always-On)}
\begin{itemize}[leftmargin=2em]
  \item Opacity separation: feasible $\alpha\approx 0.05$--$0.15$; Pareto $\alpha=1$.
  \item Size separation: feasible 1--2 px; Pareto 6--8 px.
  \item Edge/glow: Pareto only.
  \item Draw order: Pareto always on top.
  \item Persistent legends clarifying feasible vs Pareto layers.
\end{itemize}

\section{Scalability \& Performance}
\begin{itemize}[leftmargin=2em]
  \item If Dataset A is massive, default to density rendering and progressive refinement.
  \item Pareto set is never sampled.
  \item Cross-view linking updates only the highlight layer for responsiveness.
\end{itemize}

\section{Accessibility \& Clarity}
\begin{itemize}[leftmargin=2em]
  \item Do not encode $f_4$ by color alone (use shape/stroke).
  \item Heatmap palette chosen for contrast and grayscale legibility.
  \item Axis direction (min/max) indicated with small arrows or labels.
\end{itemize}

\section{Future Enhancements}
\begin{itemize}[leftmargin=2em]
  \item Knee-point detection and labeling in both spaces.
  \item Session save/load and export of views/selected solutions.
  \item What-if sliders for $(x_1,x_2)$ with live ML and objective preview.
  \item Uncertainty visualization (ML confidence or Pareto robustness).
\end{itemize}


\end{document}
